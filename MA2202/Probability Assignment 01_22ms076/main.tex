\documentclass{article}
\usepackage{amsmath}
\usepackage{amssymb}
\usepackage{amsthm}
\usepackage{graphicx} % Required for inserting images
\usepackage[margin=0.8in]{geometry} % Adjusted margins for better readability
\usepackage{enumitem} % Improved formatting for lists
%\usepackage{xcolor} % Required for defining colors
\usepackage{mdframed} % Required for framing the NOTE sections
\usepackage[usenames,dvipsnames,svgnames,table]{xcolor}
\usepackage{hyperref}
\hypersetup{
     colorlinks   = true,
     citecolor    = gray
}

\usepackage{tocloft}

\renewcommand{\cftsubsecfont}{\normalfont\hypersetup{linkcolor=purple}}
\renewcommand{\cftsubsecafterpnum}{\hypersetup{linkcolor=blue}}

% Define a color for the box
\definecolor{lightblue}{RGB}{173, 216, 230}

% Define a framed box for notes
\mdfdefinestyle{MyFrame}{%
    backgroundcolor=lightblue,
    roundcorner=5pt,
    frametitlerule=true,
    frametitlebackgroundcolor=white,
    frametitlerulecolor=lightblue,
    innertopmargin=\topskip,
}

\definecolor{lightorangered}{RGB}{255, 200, 173}

% Define a framed box for notes
\mdfdefinestyle{HW}{%
    backgroundcolor=lightorangered,
    roundcorner=5pt,
    frametitlerule=true,
    frametitlebackgroundcolor=white,
    frametitlerulecolor=lightblue,
    innertopmargin=\topskip,
}

% Define a color for the box
\definecolor{lightblue}{RGB}{173, 216, 230}

% Define a framed box for notes
\mdfdefinestyle{MyFrame}{%
    backgroundcolor=lightblue,
    roundcorner=5pt,
    frametitlerule=true,
    frametitlebackgroundcolor=white,
    frametitlerulecolor=lightblue,
    innertopmargin=\topskip,
}
% Define a framed box for notes
\mdfdefinestyle{MyFrame}{%
    backgroundcolor=lightblue,
    roundcorner=5pt,
    frametitlerule=true,
    frametitlebackgroundcolor=white,
    frametitlerulecolor=lightblue,
    innertopmargin=\topskip,
}
%permuatation and comb
\newcommand*{\permcomb}[4][0mu]{{{}^{#3}\mkern#1#2_{#4}}}
\newcommand*{\perm}[1][-3mu]{\permcomb[#1]{P}}
\newcommand*{\comb}[1][-1mu]{\permcomb[#1]{C}}
% Define a box for class session dates
\usepackage[most]{tcolorbox}

\NewTColorBox{classsessionbox}{O{}}{
    colback=white,
    colframe=lightblue,
    arc=0pt,
    outer arc=0pt,
    leftrule=0pt,
    rightrule=0pt,
    toprule=0pt,
    bottomrule=0pt,
    boxrule=0pt,
    right=0pt,
    top=5pt,
    bottom=5pt,
    width=3cm, % Set the width as needed
    title={\textbf{#1}},
    fontupper=\small, % Adjust the font size as needed
    halign=flush right, % Align to the right
}


\makeindex

% Define a theorem style
\theoremstyle{definition}
\newtheorem{theorem}{Theorem}
\usepackage{mathtools}
\DeclarePairedDelimiter\ceil{\lceil}{\rceil}
\DeclarePairedDelimiter\floor{\lfloor}{\rfloor}

\title{Probability(MA2202) Assignment 01 Solutions}
\author{Shuvam Banerji Seal (22MS076)\\ \small sbs22ms076@iiserkol.ac.in \\ GROUP - C}
\date{January 2024}
\date{\today} % Adjusted to include the current date

\begin{document}

\maketitle
\tableofcontents

\section{Problem 01:}
\begin{mdframed}[style = MyFrame]
\subsection{Problem Statement}
    Show that if you keep on throwing a fair die, the probability of eventual occurence of the outcome
six is 1.
\end{mdframed}

\subsection{Solution:}
\begin{proof}
    In the given scenario, the sample space for rolling a die once, ($\Omega_{\text{rolling a die once}}$) is $(1,2,3,4,5,6)$\\
The events will be noted as '$E_{\text{description for the event}}$'\\
So, the event of getting a six be marked as $E_6$ \\
 And the events of not getting a six be noted as $E_{\bar 6}$\\
Let $P(\text{Eventually getting a six)} > 0$ be the probability of Eventually getting a six if a die is rolled. 
    \[ \Omega_{\text{Eventually getting a six}} = \{ E_6,E_{\bar 6}E_6,E_{\bar 6}E_{\bar 6}E_6, \dots \} \cup \{E_{\bar 6}, E_{\bar 6}E_{\bar 6},E_{\bar 6}E_{\bar 6}E_{\bar 6}, \dots\} \]     
    \[ P(E_6) = p =\frac{1}{6}, \quad P(E_{\bar 6}E_6) = (1-p)p = \frac{5}{6}.\frac{1}{6}, \quad P(E_{\bar 6}E_{\bar 6}E_6) = (1-p)^2p = \frac{5}{6}.\frac{5}{6}.\frac{1}{6},\quad \dots \]
    \[\implies P( E_6,E_{\bar 6}E_6,E_{\bar 6}E_{\bar 6}E_6, \dots) = P(E_6) + P(E_{\bar 6}E_6) + \dots \]
    \[ = p + (1-p)p + (1-p)^2p + \dots \]
    \[ = p \sum_{n=0}^{\infty} (1-p)^n = \frac{p}{1-(1-p)} = \frac{\frac{1}{6}}{1-(1-\frac{1}{6})} = \frac{\frac{1}{6}}{1-\frac{5}{6}} =1\]

    Hence, $\Omega_{\text{Eventually getting a six}} = 1$
\end{proof}
\vspace{0.5cm}
\section{Problem 02:}
\begin{mdframed}[style = MyFrame]
\subsection{Problem Statement}
\[
\text{Suppose } (\Omega, \mathcal{E}, P) \text{ be a given probability space and } A_1, A_2, . . . , A_n \in \mathcal{E}. \text{ Then show that} \]
\[
P(A_1 \cup A_2 \cup . . . \cup A_n) = \sum_{i=1}^{n} P(A_i) -  \sum_{1\leq i < j\leq n} P(A_i \cap A_j) + ... + (-1)^{n-1}P(A_1\cap...\cap A_n)
\]
\end{mdframed}

\subsection{Solution:}
\begin{proof}
    
For events \(A_1, A_2, \ldots, A_n\) in a probability space \((\Omega, \mathcal{E}, P)\), the probability of their union is given by:
\[ P(A_1 \cup A_2 \cup \ldots \cup A_n) = \sum_{i=1}^{n} P(A_i) - \sum_{1 \leq i < j \leq n} P(A_i \cap A_j) + \ldots + (-1)^{n-1}P(A_1 \cap \ldots \cap A_n) \]

\textbf{Proof:}
The method to do this proof will be induction.

\textbf{Base Case (n = 2):}
\[ P(A_1 \cup A_2) = P(A_1) + P(A_2) - P(A_1 \cap A_2) \]

Since this was proved in the class, I am skipping the proof here.\\

\textbf{Inductive Step:}
Assume the inclusion-exclusion principle holds for \(n = k\):
\[ P(A_1 \cup A_2 \cup \ldots \cup A_k) = \sum_{i=1}^{k} P(A_i) - \sum_{1 \leq i < j \leq k} P(A_i \cap A_j) + \ldots + (-1)^{k-1}P(A_1 \cap \ldots \cap A_k) \]

Now, consider the case of \(n = k + 1\):
\[\therefore P(A_1 \cup A_2 \cup \ldots \cup A_k \cup A_{k+1}) \]
\[ = P(A_1 \cup A_2 \cup \ldots \cup A_k) + P(A_{k+1}) - P((A_1 \cup A_2 \cup \ldots \cup A_k) \cap A_{k+1}) \]
\\
The above was written using the principle of exclusion-inclusion only as in the base case.

\[ = \sum_{i=1}^{k} P(A_i) + P(A_{k+1}) - \sum_{1 \leq i < j \leq k} P(A_i \cap A_j) + \ldots + (-1)^{k-1}P(A_1 \cap \ldots \cap A_k)\]\[ - P((A_1 \cap A_{k+1}) \cup (A_2 \cap A_{k+1}) \cup \dots \cup (A_k \cap A_{k+1}))\]

Again expanding the last term shown here using the principle of inclusion and exclusion, we get,\\
\[ = \sum_{i=1}^{k} P(A_i) + P(A_{k+1}) - \sum_{1 \leq i < j \leq k} P(A_i \cap A_j) + \ldots + (-1)^{k-1}P(A_1 \cap \ldots \cap A_k) \]
\[ - \left( \sum_{i=1}^{k} P(A_i \cap A_{k+1}) + \ldots + (-1)^{k-1} P(A_1 \cap \ldots \cap A_k \cap A_{k+1}) \right) \]

Now, notice that the terms involving \(A_{k+1}\) cancel out:
\[ = \sum_{i=1}^{k} P(A_i) + P(A_{k+1}) - \sum_{1 \leq i < j \leq k} P(A_i \cap A_j) + \ldots + (-1)^{k} P(A_1 \cap \ldots \cap A_k) \]

This completes the more general inclusion-exclusion principle holds for all positive integers \(n\).

\end{proof}

\section{Problem 03:}


\begin{mdframed}[style = MyFrame]

\subsection{Problem Statement:}
\[
    \text{To the choice of each } n \in \mathbb{N}, \]\[\text{ could you assign a probability } P(n) > 0 \text{ such that the following conditions hold?} \]
    \[
\text{(a) } P(m) \neq P(n) \text{ for all } m, n \in \mathbb{N}. 
\]
\[
\text{(b) The probability of choosing an odd positive integer}\]
\[
\text{is the same as the probability of choosing an even positive integer.} \]
\[
\text{Justify your answer!}
\]
\end{mdframed}
\subsection{Solution:}

\begin{proof}
  
    \textbf{A take on Generalizing the problem, (For the exact example: please check after this):}
    Consider the probability distribution defined by two sequences of positive real numbers \( \{f_i\}_{i=1}^{\infty} \) and \( \{g_i\}_{i=1}^{\infty} \) such that \( \sum_{i=1}^{\infty} f_i < \infty \) and \( \sum_{i=1}^{\infty} g_i < \infty \).

    Define the probability distribution functions directly:
    \[
    P(n) = \alpha \cdot \frac{g_n}{\sum_{i=1}^{\infty} g_i} + \beta \cdot \frac{f_n}{\sum_{i=1}^{\infty} f_i}
    \]
    where $\alpha, \beta> 0$ are constants such that P(n) can modulated to $\frac{1}{2}$ for odd numbers and similarly for even numbers, and thus we can have WLOG \( \{f_i\}_{i=1}^{\infty} \) be defined as the probability of choosing each ODD number for a particular $i \in \mathbb{N}$.

    \textbf{Condition (a):} For any distinct natural numbers \( m \) and \( n \), \( P(m) \neq P(n) \).
    
    This holds because the probability distribution functions involve distinct terms for odd and even numbers.

    \textbf{Condition (b):} The probability of choosing an odd positive integer is the same as the probability of choosing an even positive integer.

    \[
    P(\text{odd}) = \alpha\frac{\sum_{n=1}^{\infty} g_n}{\sum_{i=1}^{\infty} g_i} = \frac{1}{2}
    \]
    \[
    P(\text{even}) = \beta\frac{\sum_{n=1}^{\infty} f_n}{\sum_{i=1}^{\infty} f_i} = \frac{1}{2}
    \]
for some $\alpha, \beta \in \mathbb{R}$\\
    Thus, the constructed probability distribution satisfies both conditions (a) and (b).



    \textbf{An Example:}
    Let's assign the probability of choosing even numbers sequentially be taken from the series ${f_n} = \frac{1}{2^{n+1}}$ ie for 2 the probability of choosing 2 be $\frac{1}{4}$, the probability of choosing 4 be $\frac{1}{8}$ and so on. Similarly, for the odd numbers we may take the probabilities from the terms of another convergent sequence, say $g_n = \frac{1}{3^n}$, ie probability alloted for 1 be $\frac{1}{3}$, for 3 be $\frac{1}{9}$, and so on. Hence we have,

    \[
    P(n) = 
    \begin{cases}
        \frac{1}{2^{i+1}} & \quad \text{if n is even then } i \in \mathbb{N}\\
     \frac{1}{3^i} & \quad \text{if  n is odd then } i \in \mathbb{N}
    \end{cases}
    \]


    Let's check if the given construction satisfies the provided conditions:
    \\
\textbf{(a)} For any two distinct positive integers \(m\) and \(n\), the probabilities \(P(m)\) and \(P(n)\) are different:

- If both \(m\) and \(n\) are even:
  $$ P(m) = \frac{1}{2^{a+1}} $$
  $$ P(n) = \frac{1}{2^{b+1}} $$
  Since \(a \neq b\), \(P(m) \neq P(n)\).

- If both \(m\) and \(n\) are odd:
  $$ P(m) = \frac{1}{3^a} $$
  $$ P(n) = \frac{1}{3^b} $$
  Since \(a \neq b\), \(P(m) \neq P(n)\).

- If one is even and the other is odd, the probabilities are already different.

\textbf{(b)} The probability of choosing an odd positive integer is the same as the probability of choosing an even positive integer:\\

    \[
    P(\text{Choosing an EVEN number}) = \sum_{n=1}^{\infty} \frac{1}{2^{n+1}} = \frac{\frac{1}{4}}{1 - \frac{1}{2}} = \frac{1}{2}
    \]
    \[
P(\text{Choosing an ODD number}) = \sum_{n=1}^{\infty} \frac{1}{3^{n}} = \frac{\frac{1}{3}}{1 - \frac{1}{3}} = \frac{\frac{1}{3}}{\frac{2}{3}} =\frac{1}{2}
\]








\end{proof}

\section{Problem 04:}
\begin{mdframed}[style = MyFrame]
    \subsection{Problem Statement:}
    Seven students of IISER Kolkata went to participate in an event at IISER Mohali. They booked AC 3-tier tickets from Howrah to 
 Chandigarh in Netaji Express, which has three AC 3-tier
coaches. Every such coach has eight coupes (i.e. compartments), each coupe containing eight
berths. If the berths were allocated randomly, find the probability of at least two among the
seven students being allocated berths in the same coupe
\end{mdframed}
\subsection{Solution:}
To find, \\
\[
P(\text{at least 2 students  are allotted the same compartment }) = \]
\[1 - P(\text{no 2 students  are allotted the same compartment})
\]
Let's check, as each coach has eight compartments so, total number of compartments = $3 \times 8 = 24$
\\
Going by the norms of IISER Kolkata, let's allot equal important to each student (reality may differ), then the number of possible ways of allocating different coups to the students = $\perm{24}{7}$\\

As there are 8 births in each coup, then for 7 students of equal value, allocation of birth can be done in $\perm{24}{7} \times 8^7$ waya, such that no two students are in the same coup. Again there are 192 births in all. So, the total number of ways of allocating a birth to 7 students is $\perm{192}{7}$ \\
Then using the Classical definition of probability, the possible ways of having students in different compartments is $\frac{\perm{24}{7} \times 8^7}{\perm{192}{7}}$ .\\
Finally, the probability of having at least 2 students in the same compartment is \\
\[
1 - P(\text{no 2 students  are allotted the same compartment})
\]
\[
\implies 1 - \frac{\perm{24}{7} \times 8^7}{\perm{192}{7}}
\]




\section{Problem 05:}
\begin{mdframed}[style = MyFrame]

\subsection{Problem Statement:}

    Carrom is played with a red, nine black and nine white coins (and a striker) on a square board
with a pocket in each corner. If all these coins are scattered randomly (none of them being in
any pocket) on a 29 inch × 29 inch carrom board, show that the probability of at least two of
the coins being less than three inches apart is greater than 1/2.
\end{mdframed}
\subsection{Solution:}
\begin{proof}
    
It's a 29 $\times$ 29 board meaning it's a square grid\\
We can now divide it such that the diagonal of every square $\leq 3$, i.e., two coins in the square are at least 3 inches apart\\
\[
\ceil{29}{\frac{3}{\sqrt{2}}} = \ceil{13.67}{} = 14
\]
So, we can divide the board into $14 \times 14$ grid, where the length of the diagonal is $\frac{29}{14} \times \sqrt{2} = 2.92 \leq 3$\\
Let $E_{2.92}$ denote the event where at least two coins are there whose mutual distance is $\leq 2.92$ inches, and $E_{3}$ denote the event where $\exists$ at least two cases whose mutual distance is $\leq 3$ inches, and $E_{same sq}$ denote the event where two coins are inside the same square of the $14 \times 14$ grid, which means 196 squares.\\
Hence, it can be seen that $E_{same sq} \subseteq E_{2.92} \subseteq E_3$, so if we can show that $P(E_{same sq}) \geq \frac{1}{2}$, then we are done.\\
Now, $P(E_{same sq}) \geq \frac{1}{2}$ if and only if it satisfies following relations (from Probabilistic Pigeon-Hole principle)
\[
20 - \frac{1}{2} \geq \sqrt{2 \cdot 196 \cdot \ln{\frac{1}{1 - \frac{1}{2}}} + \frac{1}{4}}
\]
\[
\implies 400 -20 + \frac{1}{4} \geq 2 \cdot 196 \cdot \ln{\frac{1}{1 - \frac{1}{2}}} + \frac{1}{4}
\]
\[
\implies
\frac{380}{2 \cdot 196} \geq \ln2
\]
\[
\implies
0.969 \geq 0.693
\]\\
which is true. So, probability of at least two of the coins being less than 3 inches apart is greater than $\frac{1}{2}$


\end{proof}
\end{document}


