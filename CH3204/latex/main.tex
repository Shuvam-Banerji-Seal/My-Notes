%%%%%%%%%%%%%%%%%%%%%%%%%%%%%%%%%%%%%%%%%%%%%%%%%%%%%%%%%
% LaTeX Beamer Presentation
% Course: CH3204 Organic Synthesis
% Title: Making Raspberry Ketone Analogue
% Theme: Metropolis (Dark)
% Version: 2 (Fixes applied based on error log)
%%%%%%%%%%%%%%%%%%%%%%%%%%%%%%%%%%%%%%%%%%%%%%%%%%%%%%%%%

\documentclass[10pt]{beamer}

%------------------------------------------------------------
% Theme and Appearance
%------------------------------------------------------------
\usetheme{metropolis} % Use the Metropolis theme (modern, dark-friendly)
\usecolortheme{default} % Or choose a specific color theme if needed

%------------------------------------------------------------
% Packages
%------------------------------------------------------------
\usepackage[utf8]{inputenc}      % Input encoding
\usepackage[T1]{fontenc}         % Font encoding
\usepackage{lmodern}             % Use Latin Modern fonts
\usepackage{graphicx}            % For including images (if needed)
\usepackage{booktabs}            % For better tables
\usepackage{chemfig}             % For drawing chemical structures
\usepackage{amsmath}             % For math formulas
\usepackage{textcomp}            % For symbols like degrees Celsius

% Biblatex for References
\usepackage[backend=biber, style=numeric, sorting=none]{biblatex}
\addbibresource{references.bib} % Name of your .bib file

%------------------------------------------------------------
% Presentation Information
%------------------------------------------------------------
\title{Making Raspberry Ketone Analogue}
\subtitle{Synthesis and Analysis of 4-(4-methoxyphenyl)butan-2-one}
\author{Your Name / Group Name}
\institute{Department of Chemistry \\ Course: CH3204 Organic Synthesis}
\date{\today}

%------------------------------------------------------------
% Start of Document
%------------------------------------------------------------
\begin{document}

%------------------------------------------------------------
% Title Frame
%------------------------------------------------------------
\begin{frame}[plain] % Plain frame for title page
    \titlepage
\end{frame}

%------------------------------------------------------------
% Outline Frame (Optional)
%------------------------------------------------------------
\begin{frame}{Outline}
    \tableofcontents
\end{frame}

%------------------------------------------------------------
% Introduction Section
%------------------------------------------------------------
\section{Introduction}

\begin{frame}{Motivation: Raspberry Ketone}
    \begin{itemize}
        \item Raspberry Ketone (4-(4-hydroxyphenyl)butan-2-one) is the primary aroma compound in red raspberries \cite{RaspberryKetoneWiki}.
        \item Widely used in flavor and fragrance industries.
        \item Potential nutraceutical properties (anti-obesity, antioxidant) \cite{PharmacologicalExplorationRK}.
        \item Structure:
            \begin{center}
                \chemfig{[:30]**6(---(-[:90]OH)-=-(-[:330]CH_2CH_2C(=[:270]O)CH_3)-=)}
                \vspace{0.5em} % Add some space
                \textit{Raspberry Ketone} % Use text instead of \caption
            \end{center}
    \end{itemize}
\end{frame}

\begin{frame}{Project Goal: Synthesizing an Analogue}
    \begin{itemize}
        \item Goal: Synthesize an analogue of Raspberry Ketone.
        \item Modification: Replace the phenolic -OH group with a methoxy (-OCH$_3$) group.
        \item Target Molecule: 4-(4-methoxyphenyl)butan-2-one (Anisylacetone).
            \begin{center}
                \chemfig{[:30]**6(---(-[:90]OCH_3)-=-(-[:330]CH_2CH_2C(=[:270]O)CH_3)-=)}
                 \vspace{0.5em} % Add some space
                 \textit{Target Analogue: Anisylacetone} % Use text instead of \caption
            \end{center}
        \item Starting Material Change: Use 4-methoxybenzaldehyde instead of 4-hydroxybenzaldehyde (due to availability).
        \item Aim: Investigate the synthesis and compare the properties of the analogue to Raspberry Ketone.
    \end{itemize}
\end{frame}

%------------------------------------------------------------
% Literature Review Section
%------------------------------------------------------------
\section{Literature Review}

\begin{frame}{Standard Synthesis of Raspberry Ketone}
    \frametitle{Established Route: Aldol Condensation + Hydrogenation}
    \begin{enumerate}
        \item \textbf{Step 1: Aldol Condensation} \cite{OnePotSynthesisRK, SynthesisRKHydrogenation}
            \begin{itemize}
                \item 4-hydroxybenzaldehyde + Acetone $\xrightarrow{\text{Base (e.g., NaOH)}}$ 4-(4-hydroxyphenyl)-3-buten-2-one (PHBA)
                \item Intermediate: $\alpha,\beta$-unsaturated ketone.
            \end{itemize}
            \begin{center} % Center the scheme
            \schemestart
                \chemfig{**6(---(-[:90]OH)-=-(-[:330]CHO)-=)}
                \+
                \chemfig{CH_3C(=[:90]O)CH_3}
                \arrow{->[Base]}[,1.5]
                \chemfig{**6(---(-[:90]OH)-=-(-[:330]CH=CHC(=[:270]O)CH_3)-=)}
            \schemestop
            \end{center}
        \item \textbf{Step 2: Catalytic Hydrogenation} \cite{SynthesisRKHydrogenation, SynthesisRKNickelBoride}
            \begin{itemize}
                \item PHBA $\xrightarrow{\text{H}_2\text{, Catalyst (e.g., Pd, Ni, Rh)}}$ Raspberry Ketone
                \item Reduction of the C=C double bond.
            \end{itemize}
            \begin{center} % Center the scheme
            \schemestart
                 \chemfig{**6(---(-[:90]OH)-=-(-[:330]CH=CHC(=[:270]O)CH_3)-=)}
                 \arrow{->[$H_2$, Catalyst]}[,1.5] % Use math mode for H_2
                 \chemfig{**6(---(-[:90]OH)-=-(-[:330]CH_2CH_2C(=[:270]O)CH_3)-=)}
            \schemestop
            \end{center}
    \end{enumerate}
\end{frame}

\begin{frame}{Starting Material: 4-Methoxybenzaldehyde}
    \frametitle{Properties of p-Anisaldehyde (C$_8$H$_8$O$_2$)}
    \begin{itemize}
        \item Common Name: p-Anisaldehyde \cite{AnisaldehydeFisherSci}.
        \item Appearance: Colorless to pale yellow liquid.
        \item Odor: Sweet, anise-like \cite{AnisaldehydeHMDB}.
        \item Key Physical Properties:
            \begin{itemize}
                \item MW: 136.15 g/mol
                \item MP: -1 to 2 \textdegree C \cite{AnisaldehydeVoloChem}
                \item BP: 247-249 \textdegree C \cite{AnisaldehydeFisherSci}
                \item Density: ~1.12 g/cm$^3$
                \item Solubility: Immiscible in water, soluble in organic solvents \cite{AnisaldehydeFisherSci}.
            \end{itemize}
        \item Structure:
            \begin{center}
                \chemfig{**6(---(-[:90]OCH_3)-=-(-[:330]CHO)-=)}
            \end{center}
    \end{itemize}
\end{frame}

\begin{frame}{Starting Material: 4-Methoxybenzaldehyde (Spectroscopy)}
    \frametitle{Expected Spectroscopic Data \cite{AnisaldehydeIRSpec, AnisaldehydeNMRSpec, AnisaldehydeHRMS}}
    \begin{columns}[T] % Split frame into columns
        \begin{column}{0.5\textwidth}
            \textbf{IR (cm$^{-1}$):}
            \begin{itemize}
                \item ~1700-1725 (C=O stretch, aldehyde)
                \item ~2700-2800 (C-H stretch, aldehyde)
                \item ~1500-1600 (C=C stretch, aromatic)
                \item ~1000-1300 (C-O stretch, methoxy)
            \end{itemize}
            \vspace{1em}
            \textbf{HRMS:}
             \begin{itemize}
                \item Expected [M+H]$^+$: 137.0597 Da
            \end{itemize}
        \end{column}
        \begin{column}{0.5\textwidth}
            \textbf{$^1$H NMR (ppm):}
            \begin{itemize}
                \item ~9.8-10.0 (s, 1H, CHO)
                \item ~6.8-7.9 (m, 4H, Ar-H)
                \item ~3.8-3.9 (s, 3H, OCH$_3$)
            \end{itemize}
            \vspace{1em}
            \textbf{$^{13}$C NMR (ppm):}
            \begin{itemize}
                \item ~190-192 (CHO)
                \item ~114-165 (Ar-C)
                \item ~55-56 (OCH$_3$)
            \end{itemize}
        \end{column}
    \end{columns}
\end{frame}

\begin{frame}{Proposed Synthesis of Analogue}
    \frametitle{Adapting the Established Route \cite{SynthesisRKNickelBoride, AnisylacetoneChemBook}}
    \begin{enumerate}
        \item \textbf{Step 1: Aldol Condensation}
            \begin{itemize}
                \item 4-methoxybenzaldehyde + Acetone $\xrightarrow{\text{Base}}$ 4-(4-methoxyphenyl)-3-buten-2-one (Anisylidene acetone)
            \end{itemize}
            \begin{center} % Center the scheme
            \schemestart
                \chemfig{**6(---(-[:90]OCH_3)-=-(-[:330]CHO)-=)}
                \+
                \chemfig{CH_3C(=[:90]O)CH_3}
                \arrow{->[Base]}[,1.5]
                \chemfig{**6(---(-[:90]OCH_3)-=-(-[:330]CH=CHC(=[:270]O)CH_3)-=)}
            \schemestop
            \end{center}
        \item \textbf{Step 2: Catalytic Hydrogenation}
            \begin{itemize}
                \item Anisylidene acetone $\xrightarrow{\text{H}_2\text{, Catalyst}}$ 4-(4-methoxyphenyl)butan-2-one (Anisylacetone)
            \end{itemize}
            \begin{center} % Center the scheme
            \schemestart
                 \chemfig{**6(---(-[:90]OCH_3)-=-(-[:330]CH=CHC(=[:270]O)CH_3)-=)}
                 \arrow{->[$H_2$, Catalyst]}[,1.5] % Use math mode for H_2
                 \chemfig{**6(---(-[:90]OCH_3)-=-(-[:330]CH_2CH_2C(=[:270]O)CH_3)-=)}
            \schemestop
            \end{center}
    \end{enumerate}
    This two-step sequence is expected to be the most direct route.
\end{frame}

%------------------------------------------------------------
% Experimental Section (FILL THIS IN WITH YOUR DETAILS)
%------------------------------------------------------------
\section{Experimental Procedure}

\begin{frame}{Step 1: Aldol Condensation - Procedure}
    \frametitle{Synthesis of 4-(4-methoxyphenyl)-3-buten-2-one}
    \begin{alertblock}{Your Experimental Details Needed}
        Describe the exact procedure you followed:
        \begin{itemize}
            \item Reagents used (4-methoxybenzaldehyde, acetone, base - specify type and concentration).
            \item Amounts/Moles of each reagent.
            \item Solvent used (if any).
            \item Reaction conditions (temperature, time).
            \item Observations during the reaction (color changes, precipitation).
            \item Work-up procedure (quenching, extraction, washing, drying).
            \item Isolation/Purification method (e.g., recrystallization, chromatography - specify details).
        \end{itemize}
    \end{alertblock}
\end{frame}

\begin{frame}{Step 2: Hydrogenation - Procedure}
    \frametitle{Synthesis of 4-(4-methoxyphenyl)butan-2-one}
     \begin{alertblock}{Your Experimental Details Needed}
        Describe the exact procedure you followed:
        \begin{itemize}
            \item Starting material (isolated intermediate or one-pot?).
            \item Hydrogenation catalyst used (e.g., Pd/C, Ni$_2$B, Raney Ni - specify amount).
            \item Hydrogen source (H$_2$ gas pressure, transfer hydrogenation?).
            \item Solvent used.
            \item Reaction conditions (temperature, time, pressure).
            \item Monitoring the reaction (e.g., TLC, GC).
            \item Work-up procedure (catalyst filtration, solvent removal).
            \item Purification method (e.g., distillation, chromatography - specify details).
        \end{itemize}
    \end{alertblock}
\end{frame}

%------------------------------------------------------------
% Results and Analysis Section (FILL THIS IN)
%------------------------------------------------------------
\section{Results and Analysis}

\begin{frame}{Analysis of Intermediate (Anisylidene acetone)}
    \frametitle{Characterization of 4-(4-methoxyphenyl)-3-buten-2-one}
     \begin{alertblock}{Your Results Needed}
        Present the data obtained for the intermediate (if isolated/analyzed):
        \begin{itemize}
            \item Appearance (color, physical state).
            \item Yield (mass, percentage).
            \item Melting Point (if solid, compare to literature if available).
            \item TLC data (Rf value, solvent system).
            \item IR Spectrum: Assign key peaks (C=O, C=C, C-O). Compare to literature/expected.
            \item $^1$H NMR Spectrum: Assign key peaks (vinyl H, Ar-H, OCH$_3$, CH$_3$). Compare to literature/expected.
            \item $^{13}$C NMR Spectrum (if obtained): Assign key peaks.
            \item Other analyses performed?
        \end{itemize}
     \end{alertblock}
     \textit{Note: Literature MP for Anisylidene acetone varies, often around 72-75 \textdegree C.}
\end{frame}

\begin{frame}{Analysis of Final Product (Anisylacetone)}
    \frametitle{Characterization of 4-(4-methoxyphenyl)butan-2-one}
     \begin{alertblock}{Your Results Needed}
        Present the data obtained for the final product:
        \begin{itemize}
            \item Appearance (color, physical state - likely an oil).
            \item Yield (mass, overall percentage yield from starting aldehyde).
            \item Melting Point (Literature: 8-10 \textdegree C \cite{AnisylacetoneChemBook} - likely liquid at room temp).
            \item Boiling Point (if distilled, compare to literature: ~270 \textdegree C atm \cite{AnisylacetoneChemImpex}, lower under vacuum).
            \item TLC data (Rf value, solvent system).
            \item IR Spectrum: Assign key peaks (C=O ketone, C-O ether). Compare to expected (absence of C=C, presence of C-O).
            \item $^1$H NMR Spectrum: Assign key peaks (CH$_2$, Ar-H, OCH$_3$, CH$_3$). Compare to expected.
            \item $^{13}$C NMR Spectrum (if obtained): Assign key peaks. Compare to expected.
            \item GC-MS / HRMS data (if obtained): Confirm mass, purity.
        \end{itemize}
     \end{alertblock}
\end{frame}

\begin{frame}{Comparison: Analogue vs. Raspberry Ketone}
    \frametitle{Spectroscopic and Physical Property Differences}
    Based on literature \cite{SynthesisRKNickelBoride, AnisylacetoneChemBook, AnisylacetoneChemImpex, RKPropertiesChemImpex} and expected data:
    \begin{table}[h]
        \centering
        \caption{Comparison of Key Properties}
        \begin{tabular}{lcc}
            \toprule
            Property & Raspberry Ketone & Anisylacetone (Analogue) \\
            \midrule
            Structure & Ar-OH & Ar-OCH$_3$ \\ % Simplified representation
                      & (Ar = p-C$_6$H$_4$CH$_2$CH$_2$COCH$_3$) & (Ar = p-C$_6$H$_4$CH$_2$CH$_2$COCH$_3$) \\ % Define Ar group
            MW (g/mol) & 164.20 & 178.23 \\
            MP (\textdegree C) & 82-83 & 8-10 (oil at RT) \\
            IR: O-H stretch & Present (~3370 cm$^{-1}$) & Absent \\
            IR: C-O-C stretch & Absent & Present (~1000-1300 cm$^{-1}$) \\
            $^1$H NMR: Phenolic OH & Present (~6-7 ppm) & Absent \\
            $^1$H NMR: Methoxy CH$_3$ & Absent & Present (~3.8 ppm) \\
            Odor & Raspberry-like & Sweet, floral, fruity \\
            \bottomrule
        \end{tabular}
        \vspace{1em}
        \textit{Our experimental results should be compared to these expected values.}
    \end{table}
\end{frame}

%------------------------------------------------------------
% Conclusion Section
%------------------------------------------------------------
\section{Conclusion}

\begin{frame}{Conclusion}
     \begin{alertblock}{Summarize Your Findings}
        Conclude the presentation:
        \begin{itemize}
            \item Briefly restate the goal (synthesize Raspberry Ketone analogue).
            \item Summarize the synthetic route used (Aldol + Hydrogenation with 4-methoxybenzaldehyde).
            \item State whether the synthesis was successful based on your analytical data.
            \item Mention the overall yield obtained.
            \item Highlight key characterization results confirming the structure (mention specific IR, NMR evidence).
            \item Briefly compare the properties of your synthesized analogue to literature values and to Raspberry Ketone (e.g., physical state difference due to H-bonding).
            \item Mention any challenges encountered during the synthesis or analysis.
            \item Suggest potential future work (if any).
        \end{itemize}
     \end{alertblock}
\end{frame}

%------------------------------------------------------------
% References Section
%------------------------------------------------------------
\section*{References} % Use section* for unnumbered section in ToC

\begin{frame}[allowframebreaks]{References} % allowframebreaks allows list to span multiple slides
    \printbibliography
\end{frame}

%------------------------------------------------------------
% End of Document
%------------------------------------------------------------
\end{document}
%%%%%%%%%%%%%%%%%%%%%%%%%%%%%%%%%%%%%%%%%%%%%%%%%%%%%%%%%
